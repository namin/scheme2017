% Created 2017-05-26 Fri 18:39
% Intended LaTeX compiler: pdflatex
\documentclass[acmlarge]{acmart}


\acmConference{Scheme and Functional Programming Workshop 2017}{September 03, 2017}{Oxford, UK}
\acmSubmissionID{}
\acmDOI{}
\setcopyright{none}
\copyrightyear{2017}
\settopmatter{printacmref=false, printccs=false, printfolios=false}
\acmVolume{}
\acmNumber{}
\acmArticle{2}
\acmYear{2017}
\acmMonth{9}
\renewcommand\footnotetextcopyrightpermission[1]{}
\usepackage{fancyhdr}
\pagestyle{fancy}
\fancyfoot{}
\fancyfoot[R]{}

\usepackage[utf8]{inputenc}
\usepackage[T1]{fontenc}
\usepackage{graphicx}
\usepackage{grffile}
\usepackage{longtable}
\usepackage{wrapfig}
\usepackage{rotating}
\usepackage[normalem]{ulem}
\usepackage{amsmath}
\usepackage{textcomp}
\usepackage{amssymb}
\usepackage{capt-of}
\usepackage{hyperref}
\author{Joseph Corneli and Raymond Puzio}
\date{26 May, 2017}
\title[Talk]{Extending the LISP model: from cons cells to triples, from trees to hypergraphs}
\hypersetup{
 pdfauthor={Joseph Corneli and Raymond Puzio},
 pdftitle={Extending the LISP model: cons cells to triples, trees to hypergraphs},
 pdfkeywords={arxana, hypertext, inference anchoring},
 pdfsubject={Organizer for presentation on arxana and math text analysis at Oxford.},
 pdfcreator={Emacs 25.1.50.2 (Org mode 9.0.5)}, 
 pdflang={English}}
\begin{document}

\maketitle
\thispagestyle{fancy}

Arxana is a higher-dimensional variant of LISP, based on
nested semantic networks instead of cons cells.  In contradistinction
to LISP where the fundamental building block is a cell
`\verb|(a . b)|', Arxana's fundamental building block is a triple,
`\verb|(a c b)|'.  Furthermore, in the language of the Semantic Web,
every triple is `reified'.  Links and their constituent positions can
contain further structure or be augmented with offset annotations: for
example, we distinguish between `\verb|((a d e) c b)|' and
`\verb|(a c b)| \(\oplus_1\) \verb|(a d e)|'.  The first form models
an assertion about the link `\verb|(a d e)|', and the second models an
assertion about the atom `\verb|a|' within `\verb|(a c b)|'.  These
facilities allow us to build, reason about, query, and program with
hypergraphs rather than trees or networks.  This representation
strategy is useful for building runnable conceptual models of complex
and recursive structures.  Modelling informal mathematical discourse is a
motivating application: this requires a different approach from
the strictly deductive style of formal mathematics.  Other programming
languages which support a similar annotative style include Kurzweil's
\emph{Flare} and Nelson's \emph{ZigZag}.  Our Arxana prototypes are
implemented in Emacs Lisp.
\end{document}
