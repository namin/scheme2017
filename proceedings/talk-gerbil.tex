\documentclass[acmlarge]{acmart}

%\settopmatter{printacmref=false}
\settopmatter{}

\acmDOI{}


\acmConference{Scheme and Functional Programming Workshop 2017}{September 03, 2017}{Oxford, UK}
\acmSubmissionID{}
\acmDOI{}
\setcopyright{none}
\copyrightyear{2017}
\settopmatter{printacmref=false, printccs=false, printfolios=false}
\acmVolume{}
\acmNumber{}
\acmArticle{4}
\acmYear{2017}
\acmMonth{9}
\renewcommand\footnotetextcopyrightpermission[1]{}
\usepackage{fancyhdr}
\pagestyle{fancy}
\fancyfoot{}
\fancyfoot[R]{}

\title[Talk]{Gerbil on Gambit, as they say Racket on Chez}
\author{Dimitris Vyzovitis}
\email{vyzo@hackzen.org}

\begin{document}
\maketitle
\thispagestyle{fancy}

Gerbil is a new opinionated dialect of Scheme designed for Systems
Programming with a state of the art macro and module system on top of
the Gambit runtime. Gerbil wants to fill the the needs of seasoned
Schemers and Common LISP refugees who want to do their systems
programming with modern macro facilities and without loss of
performance.

The system implements modularity and language extensibility facilities
equivalent to Racket's, including the {\tt \#lang} reader.  Gerbil implements
this with a macro expander, compiler, and standard library sitting on
top of Gambit.  As such, it brings these facilities for the first time
to a highly performant Ahead of Time natively compiled environment.

The relationship between Gerbil and Gambit is symbiotic: Gerbil
provides the top-half and Gambit the bottom-half ot the system,
similar to how Racket will run on Chez in the not so distant future.

\vspace{1cm}
Source code: \url{https://github.com/vyzo/gerbil}

\end{document}
