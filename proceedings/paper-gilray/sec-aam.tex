

\section{Control-flow Analysis} 
\label{sec:aam}
%
This section introduces control-flow analysis by instantiating it for the continuation-passing-style (CPS)
$\lambda$-calculus. 
%
We follow the abstracting abstract machines (AAM) methodology, a systematic process for developing a static analysis
(an approximating semantics) from a precise operational semantics of an abstract machine.


Static analysis by abstract interpretation proves properties of a program by running code through an interpreter powered by an 
\textit{abstract semantics} that approximates the behavior of an exact \textit{concrete semantics}.
%
This process is a general method for analyzing programs and serves applications such as program verification, malware/vulnerability detection, and
compiler optimization, among others 
\cite{cousot:1976:staticdetermination,cousot:1977:unifiedlatticemodel,cousot:1979:systematicdesign,midtgaard:2012:cfa}.
%
Van Horn and Might's approach of \textit{abstracting abstract machines} (AAM) uses abstract interpretation of abstract machines for \textit{control-flow analysis} (CFA) of functional (higher-order) 
programming languages \cite{johnson:2013:oaam,might:2010:aam,might:2010:free}.
%
The AAM methodology is flexible and allows a high degree of control over how program states are represented.
%
AAM provides a general method for automatically abstracting an arbitrary small-step 
abstract-machine semantics to obtain an approximation in a variety of styles.
%
Importantly, one such style aims to focus all unboundedness in a semantics on the machine's address-space.
%
This makes the strategy used for the allocation of addresses crucial to the tradeoff struck between precision and complexity~\cite{gilray:2016:allocation}, and results in a highly flexible and tunable analysis infrastructure.
%
More broadly, the approach has been used to instantiate both traditional finite flow analyses and heavy-weight program verification \cite{nguyen:2014:SCV}. 


\subsection{A Concrete Operational Semantics}
\label{sec:aam:concrete}
%
This section reviews the process of producing a formal operational semantics for a simple language \cite{plotkin:1981:structural},
specifically, the untyped $\lambda$-calculus in \textit{continuation-passing style} (CPS).
%
CPS constrains call sites to tail position so that functions may never return; instead, callers explicitly pass a continuation forward to be invoked on the return value \cite{plotkin:1975:cps}.
%
This makes our semantics tail recursive (small-step) and easier to abstract while entirely eliding the challenges of manually managing a stack and its abstraction, a process previously discussed in the context of AAM \cite{might:2010:aam, johnson:2014:aac}.
%
Using an AAM that explicitly models the stack in a precise manner, while allowing for adjustable allocation, has also been recently addressed \cite{gilray:2016:p4f}.


The grammar structurally distinguishes between call-sites $call$ and atomic-expressions $ae$:
%
\begin{align*}
	\var{call} \in \syn{Call} ::=&\ \ttlp \var{ae}\ \var{ae}\ \ldots \ttrp\ \vert\ \ttlp \mathtt{halt} \ttrp  
  \\
	\var{lam} \in \syn{Lam} ::=&\ \ttlp \lambda\ \ttlp x\ \ldots \ttrp\ \var{call} \ttrp 
  \\
	\var{ae} \in \syn{AE} ::=&\ \var{lam}\ \vert\ x  
  \\
	x \in \syn{Var} \textit{\ is}&\textit{\ a set of program variables}
\end{align*}
%
Instead of specifically affixing each expression with a unique label, we assume two identical expressions occurring separately in a program are not equal.
%
While a direct-style language with a variety of continuations (e.g., argument continuations, \texttt{let}-continuations, etc.), 
or extensions such as recursive-binding forms, conditionals, mutation, or primitive operations, 
would add complexity to any semantics, they do not affect the concepts we are exploring and so are left out.


We define the evaluation of programs in this language using a relation ($\cTo$), over states of an abstract-machine, which determines how the 
machine transitions from one state to another.
%
States $(\varsigma)$ range over control expression (a call site), binding environment, and value store components:
%
\begin{align*}
  \varsigma \in \Sigma \defas&\ \syn{Call} \times \var{Env} \times \var{Store} 
  \\
  \rho \in \var{Env} \defas&\ \syn{Var} \rightharpoonup \var{Addr}  
  \\
  \sigma \in \var{Store} \defas&\ \var{Addr} \rightharpoonup \var{Value}  
  \\
  a \in \var{Addr} \defas&\ \syn{Var} \times \mathbb{N} 
  \\
  v \in \var{Value} \defas&\ \var{Clo}
  \\
  \var{clo} \in \var{Clo} \defas&\ \syn{Lam} \times \var{Env} 
\end{align*}
%
Environments $(\rho)$ map variables in scope to an address for the visible binding.
%
Value stores $(\sigma)$ map these addresses to values (in this case, closures); these may be thought of as a model of the heap.
%
Both these functions are partial and accumulate points as execution progresses.


Evaluation of atomic expressions is handled by an auxiliary function $(\mathcal{A})$ which produces a value $(\var{clo})$
for an atomic expression in the context of a state $(\varsigma)$.
%
This is done by a lookup in the environment and store for variable references $(x)$, and by closure creation for $\lambda$-abstractions $(\var{lam})$.
%
In a language containing syntactic literals, these would be translated into equivalent semantic values here.
%
\begin{align*}
  \mathcal{A} : \mathsf{AE} \times \Sigma \rightharpoonup&\ \var{Value} 
  \\
  \mathcal{A}(x,\ (\var{call},\ \rho,\ \sigma)) \defas&\ \sigma(\rho(x)) 
  \\
  \mathcal{A}(\var{lam},\ (\var{call},\ \rho,\ \sigma)) \defas&\ (\var{lam},\ \rho) 
\end{align*}


The transition relation $(\cTo) : \Sigma \rightharpoonup \Sigma$ yields at most one successor for a given predecessor in the state-space $\Sigma$.
%
This is defined:
%
\begin{align*} 
  \overbrace{(\ttlp \var{ae}_f\ \var{ae}_1\ \ldots\ \var{ae}_j \ttrp, \rho, \sigma)}^{\varsigma} 
  \cTo 
  (\var{call}', \rho', \sigma')
\end{align*}
\vspace{-0.5cm}
\begin{align*}
  \textrm{where} \ \ \ \ \ \ & (\lamform{x_0 \ldots x_j}{\var{call}'}, \rho_\lambda) = \mathcal{A}(\var{ae}_f,\ \varsigma) 
  \\
  & v_i = \mathcal{A}(\var{ae}_i,\ \varsigma)  
  \\
  & \rho' = \rho_\lambda [ x_i \mapsto a_i ]
  \\
  & \sigma' = \sigma [ a_i \mapsto v_i ]
  \\
  & a_i = (x_i, \vert \var{dom}(\sigma) \vert)
\end{align*}
%
Execution steps to the call-site body of the lambda invoked (as given by the atomic-evaluation of $\var{ae}_f$).
%
This closure's environment $(\rho_\lambda)$ is extended with a binding for each variable $x_i$ to a fresh address $a_i$.
%
A particular strategy for allocating a fresh address is to pair the variable being allocated for with the current number of points in the store (a value that increases after each set of new allocations).
%
The store is extended with the atomic evaluation of $\var{ae}_i$ for each of these addresses $a_i$.
%
A state becomes stuck if \texttt{(halt)} is reached or if the program is malformed (\textit{e.g.}, a free variable is encountered).


To fully evaluate a program $\var{call}_0$ using these transition rules, we \textit{inject} it into our state space using a helper 
$\mathcal{I} : \syn{Call} \to \Sigma$:
%
\begin{align*}
  \mathcal{I}(\var{call}) \defas (\var{call}, \varnothing, \varnothing)
\end{align*}


We may now perform the standard lifting of $(\cTo)$ to a collecting semantics defined over sets of states:
%
\begin{align*}
  s \in S \defas \mathcal{P}(\Sigma)
\end{align*}
%
Our collecting relation $(\cToW)$ is a monotonic, total function that gives a set including the trivially reachable state $\mathcal{I}(\var{call}_0)$
plus the set of all states immediately succeeding those in its input.
%
\begin{align*}
  s \cToW \Set{ \varsigma'\ \vert\ \varsigma \in s \wedge \varsigma \cTo \varsigma' } \cup \Set{ \mathcal{I}(\var{call}_0) }
\end{align*}


If the program $\var{call}_0$ terminates, iteration of $(\cToW)$ from $\bot$ (i.e., the empty set $\varnothing$) does as well.
%
That is, ${(\cToW)}^{n}(\bot)$ is a fixed point containing $\var{call}_0$'s full program trace for some $n \in \mathbb{N}$ whenever $\var{call}_0$ is a terminating program.
%
No such $n$ is guaranteed to exist in the general case (when $\var{call}_0$ is a non-terminating program)
as our language (the untyped CPS $\lambda$-calculus) is Turing-equivalent, our semantics is fully precise, and the state-space we defined is infinite.


\subsection{An Abstract Operational Semantics}
\label{sec:aam:abstract}
%
Now that we have formalized program evaluation using our concrete semantics as iteration to a (possibly infinite) fixed point, we are ready to design a 
computable approximation of this fixed point (the exact program trace) using abstract interpretation.
%
Previous work has explored a wide variety of approaches to systematically abstracting a semantics like these 
\cite{might:2010:aam,johnson:2013:oaam,might:2010:free}.
%%%% ^^ more references here ^^
%
Broadly construed, the nature of these changes is to simultaneously finitize the domains of our machine while introducing non-determinism both into the 
transition relation (multiple successor states may immediately follow a predecessor state) and the store (multiple values may become conflated at a single address).
%
We use a finite address space to cut the otherwise mutually recursive structure of values (closures) and environments.
%
(Without addresses and a value store, environments map variables directly to closures and closures contain environments). 
% 
A finite address space yields a finite state space overall and ensures the computability of our analysis.
%
Typographically, components unique to this \textit{abstract} abstract machine wear hats so we can tell them apart without confusing essential underlying roles:
%
\begin{align*}
  \hat{\varsigma} \in \hat{\Sigma} \defas&\ \mathsf{Call} \times \widehat{\var{Env}} \times \widehat{\var{Store}}
  \\
  \hat{\rho} \in \widehat{\var{Env}} \defas&\ \mathsf{\var{Var}} \rightharpoonup \widehat{\var{Addr}}  
  \\
  \hat{\sigma} \in \widehat{\var{Store}} \defas&\ \widehat{\var{Addr}} \rightarrow \widehat{\var{Value}}  
  \\
  \hat{a} \in \widehat{\var{Addr}} \defas&\ \mathsf{Var}
  \\
  \hat{v} \in \widehat{\var{Value}} \defas&\ \mathcal{P}(\widehat{\var{Clo}})  
  \\
  \widehat{\var{clo}} \in \widehat{\var{Clo}} \defas&\ \mathsf{Lam} \times \widehat{\var{Env}} 
\end{align*}
%
Value stores are now total functions mapping abstract addresses to a \textit{flow set} ($\hat{v}$) of zero or more abstract closures. 
%
This allows a range of values to merge and inhabit a single abstract address, introducing imprecision into our abstract semantics, but also allowing
for a finite state space and a guarantee of computability.
%
To begin, we use a monovariant address set $\widehat{\var{Addr}}$ with a single address for each syntactic variable.
%
This choice (and its alternatives) is at the heart of our present topic and will be returned to shortly.



Evaluation of atomic expressions is handled by an auxiliary function $(\hat{\mathcal{A}})$ which produces a flow set $(\hat{v})$
for an atomic expression in the context of an abstract state $(\hat{\varsigma})$.
%
In the case of closure creation, a singleton flow set is produced.
%
\begin{align*}
  \hat{\mathcal{A}} : \mathsf{AE} \times \hat{\Sigma} \rightharpoonup&\ \widehat{\var{Value}} 
  \\
  \hat{\mathcal{A}}(x,\ (\var{call},\ \hat{\rho},\ \hat{\sigma})) \defas&\ \hat{\sigma}(\hat{\rho}(x)) 
  \\
  \hat{\mathcal{A}}(\var{lam},\ (\var{call},\ \hat{\rho},\ \hat{\sigma})) \defas&\ \Set{(\var{lam},\ \hat{\rho})} 
\end{align*}


The abstract transition relation $(\pdTo) \subseteq \hat{\Sigma} \times \hat{\Sigma}$ yields any number of successors for a given predecessor in the state-space $\hat{\Sigma}$.
%
As mentioned when introducing AAM, there are two fundamental changes required using this approach. 
%
Because abstract addresses can become bound to multiple closures in the store and atomic evaluation produces a flow set containing zero or more closures, one successor state
results for each closure bound to the address for $\var{ae}_f$.
%
Also, due to the relationality of abstract stores, we can no longer use strong update when extending the store $\hat{\sigma}'$.
%
\begin{align*} 
  \overbrace{(\sexpr{\var{ae}_f\ \var{ae}_1\ \ldots\ \var{ae}_j}, \hat{\rho}, \hat{\sigma})}^{\hat{\varsigma}} 
  \pdTo 
  (\var{call}', \hat{\rho}', \hat{\sigma}')
\end{align*}
\vspace{-0.5cm}
\begin{align*}
  \textrm{where} \ \ \ \ \ \ & (\lamform{x_0 \ldots x_j}{\var{call}'}, \hat{\rho}_\lambda) \in \hat{\mathcal{A}}(\var{ae}_f,\ \hat{\varsigma}) 
  \\
  & \hat{v}_i = \hat{\mathcal{A}}(\var{ae}_i,\ \hat{\varsigma})  
  \\
  & \hat{\rho}' = \hat{\rho}_\lambda [ x_i \mapsto \hat{a}_i ]
  \\
  & \hat{\sigma}' = \hat{\sigma} \sqcup [ \hat{a}_i \mapsto \hat{v}_i ]
  \\
  & \hat{a}_i = x_i
\end{align*}
%
A weak update is performed on the store instead which results in the least upper bound of the existing store and each new binding.
%
Join on abstract stores distributes point-wise:
%
\begin{align*}
  \hat{\sigma} \sqcup \hat{\sigma}' &\defas \lambda \hat{a}.\ \hat{\sigma}(\hat{a}) \cup \hat{\sigma}'(\hat{a})
\end{align*}
%
Unless it is desirable, and provably safe to do so \cite{might:2006:gammacfa}, we never remove closures already seen.
%
Instead, we strictly accumulate every closure bound to each $\hat{a}$ (i.e., abstract closures which simulate closures bound to addresses which $\hat{a}$ simulates)
over the lifetime of the program.
%
A flow set for an address $\hat{a}$ indicates a range of values which over-approximates all possible concrete values that can flow to any 
concrete address approximated by $\hat{a}$.
%
For example, if a concrete machine binds $(y, 345) \mapsto \var{clo}_1$ and $(y, 903) \mapsto \var{clo}_2$, 
its monovariant approximation might bind $y \mapsto \{ \widehat{\var{clo}}_1, \widehat{\var{clo}}_2 \}$.
%
Precision is lost for $(y, 345)$ both because its value has been merged with $\widehat{clo}_2$, and because the environments for $\widehat{clo}_1$ and 
$\widehat{clo}_2$ in-turn generalize over many possible addresses for their free variables (the environment in $\widehat{\var{clo}}_1$ is less precise than that in $\var{clo}_1$).


To approximately evaluate a program according to these abstract semantics, we first define an abstract injection function,
$\hat{\mathcal{I}}$, where the store begins as a function, $\bot$, that maps every abstract address to the empty set.
%
\begin{align*}
 \hat{\mathcal{I}} :&\ \syn{Call} \to \hat{\Sigma}
 \\
 \hat{\mathcal{I}}(\var{call}) \defas&\ (\var{call}, \varnothing, \bot)
\end{align*}


We again lift $(\pdTo)$ to obtain a collecting semantics $(\pdToS)$ defined over sets of states:
%
\begin{align*}
  \hat{s} \in \hat{S} &\defas \Pow{\hat{\Sigma}}
\end{align*}
%
Our collecting relation $(\pdToS)$ is a monotonic, total function that gives a set including the trivially reachable finite-state $\hat{\mathcal{I}}(\var{call}_0)$
plus the set of all states immediately succeeding those in its input. 
%
\begin{align*}
  \hat{s} \pdToS \hat{s}', \text{where}
\end{align*}
\vspace{-0.6cm}
\begin{align*}
  %\text{where}\ \ \ \ \ \ 
  \hat{s}' = \Set{ \hat{\varsigma}'\ \vert\ \hat{\varsigma} \in \hat{s} \wedge \hat{\varsigma} \pdTo \hat{\varsigma}' } \cup \Set{ \hat{\mathcal{I}}(\var{call}_0) }
\end{align*}
%
Because $\widehat{Addr}$ (and thus $\hat{\Sigma}$) is now finite, we know the approximate evaluation of even a non-terminating $\var{call}_0$ will terminate.
%
That is, for some $n \in \mathbb{N}$, the value $(\pdToS)^{n}(\bot)$ is guaranteed to be a fixed point containing an approximation of $\var{call}_0$'s full concrete program trace
\cite{tarski:1955:fixpoint}.



\subsubsection{Widening and Extension to Larger Languages}
\label{sec:aam:extension}
%
Various forms of widening and further approximations may be layered on top of the na\"ive analysis ($\pdToS$).
%
One such approximation is store widening, which is necessary for our analysis to be polynomial-time in the size of the program.
%
This structurally approximates the analysis above, where each state contains a whole store, by pairing a set of states without stores, with a single, global store that overapproximates all possible bindings.
%
This global store is maintained as the least-upper-bound of all bindings that are encountered in the course of analysis.



Setting up a semantics for real language features such as conditionals, primitive operations, direct-style recursion, 
or exceptions, is no more difficult, if more verbose.
%
Supporting direct-style recursion, for example, requires an explicit stack as continuations are no longer baked into the 
source text by CPS conversion.
%
Handling other forms is often as straightforward as including an additional transition rule for each.
%

