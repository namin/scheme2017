

\section{Datalog-based Flow Analysis}
%
In this section, we will give an overview of Datalog as a declarative logic-programming
language, show how it may be executed using a bottom-up fixed-point algorithm, and discuss
our encoding of abstract semantics from the previous section.
%



\subsection{Datalog}
%
A datalog program consists of a set of relations, along with the \textit{rules}
pertaining to them.
%
A relation encodes a set of tuples known as \textit{facts}.
%
For example, if we have a relation $\var{Parent}$, the fact $\var{Parent}(p,c)$
may assert that $p$ is a parent of $c$.
%
A rule then takes the form $a_0\ :\!-\ a_1, \ldots, a_j$ where a comma denotes conjunction and
each atom $a_i$ is of the form $r(x, \ldots)$, where $r$ is a relation and each $x$ is a variable.
%
In Datalog the turnstile means ``is implied by''; this is because the rules are formally an implication
form of a Horn clause.
%
Horn clauses are disjunctions where all but one of the atoms are negated: $a_0 \vee \neg a_1 \vee \ldots \vee \neg a_j$.
%
This is the same as $a_0 \vee \neg (a_1 \wedge \ldots \wedge a_j)$ by De Morgan's laws, which is the same as an implication:
$a_0 \impliedby a_1 \wedge \ldots \wedge a_j$.
%
%The satisfiability (SAT) problem for conjunctions of horn clauses (or datalog rules) is complete for PTIME \cite{}. 
%
For example, a rule for computing grandparents can be specified:
%
\[
  \var{GrandParent}(gp, c)\ :\!-\ Parent(gp, p), Parent(p, c).
\]





\subsection{Datalog solvers}
%
A number of strategies exist for executing a datalog program---that is, finding a sound least-fixed-point
that contains all the input facts and obeys every rule.
%
Unlike solvers for Prolog and other logic programming systems, bottom-up approaches have tended to prevail
\cite{ullman:1989:bottomup} although this can depend on extensions to the language and its use.
%
Bottom-up solving begins with the set of facts provided as input, and iterates a monotonic function
encoding the set of rules until a least-fixed-point is reached.
%
The data structure for relations and the operation of these increasing iterations varies by approach.


The datalog solver \textit{bddbddb} uses binary decision diagrams (BDDs) to encode relations and supports
efficient relational algebra (RA) over these structures \cite{whaley:2005:bddbddb}.
%
BDDs are decision trees that encode arbitrary relations by viewing each bit in the input as a binary decision.
%
They have the potential to be a compact and efficient representation of a relation, but have a worst-case
exponential space complexity and are highly sensitive to variable ordering (in what order are bits of input
branched on).


The datalog solver Souffl\'e uses the semi-na\"ive bottom up algorithm with partial-evaluation-based optimization
\cite{scholz:2016:souffle}.
%
A relational algebra machine (RAM) executes the program in bottom-up fashion by using efficient underlying data
structures for relations (such as prefix trees) and directly selects and propagates tuples in a series of nested loops.
%
This RAM then has a partial-evaluation applied to it that optimizes the interpretation for the specific set of
datalog rules.
%
The tool accepts a datalog program as input, performs this partial evaluation and writes a C++ program that
parallelizes the outside loops of each RA operation using pthreads.



\subsection{Encoding CFA}
%
Various program analysis have been implemented using Datalog. Both bddbddb and Souffl\'e presented
program analyses in their experiments. Another prominent example is the DOOP framework for Java
\cite{smaragdakis:2011:pickyourcontexts}, used to demonstrate a generalization of object-sensitive flow analysis.


To encode a control-flow analysis as a Datalog problem, we first represent the abstract syntax tree (AST) as
a set of relations where each expression and variable in the program has been enumerated.
%
In the souffl\'e syntax, the encoding for Lambda expressions, conditional expressions, and Variable references is as follows:
%
\begin{Verbatim}
  .decl SyntaxLambdaExp(e:Exp, x:Var, ebody:Exp) input
  .decl SyntaxIfExp(e:Exp, x:Var, e0:Exp, e1:Exp) input
  .decl SyntaxVarExp(e:Exp, x:Var) input
\end{Verbatim}
%
A tuple $(e_0,x,\var{e_1})$ in the \texttt{SyntaxLambdaExp} relation indicates that the expression $e_0$ is a lambda
with the formal parameter $x$ and the expression body $e_1$. A tuple $(e_0,x,e_1,e_2)$ in the \texttt{SyntaxIfExp} relation
indicates that the expression $e_0$ is a conditional branching on the variable $x$ with the true branch $e_1$ and false
branch $e_2$. A tuple $(e,x)$ in the \texttt{SyntaxVarExp} relation indicates that the expression $e$ is a variable reference
returning the value of variable $x$.


We then encode the constraints of our analysis directly as rules in datalog, such as:
%
\begin{Verbatim}
StoreEdge(x,y) :-
  SyntaxVarExp(e,x),
  ReachableExp(e,kaddr),
  KStore(kaddr,y,ebody,kaddr0).
\end{Verbatim}
%
This rule adds a flow from $x$ to $y$ when a reachable expression $e$ is returning the value at $x$ and
its continuation binds the variable $y$.

