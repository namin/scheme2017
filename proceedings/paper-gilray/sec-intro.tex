
\section{Introduction}
%
A \textit{control-flow analysis} (CFA) of a functional programming language models the
propagation of data flows and control flows through a target program, across all
possible executions.
%
In static analyses of functional languages generally, a model of where functional values
(i.e., lambdas, closures) can flow is required to model where control can flow
from a call site, and vice versa.
%
At a call site in Scheme, \texttt{(f x)}, the possible values for \texttt{f} determine
which lambda bodies can be reached.
%
Likewise, the possible callers for the lambda that binds \texttt{f},
$\texttt{(lambda ($\ldots$ f $\ldots$) $\ldots$)}$, influence the lambdas that
can flow into \texttt{f}.
%
This mutual dependence of data flow and control flow can be handled using an abstract interpretation,
simultaneously modeling all interdependent language features.
%
There are systematic approaches to designing analyses such as these, however the traditional worklist algorithms
used to implement them in practice are inefficient and have difficulty scaling.
%
Even with optimizations such as global-store widening and flat environments, the analysis is in $O(n^3)$ in the
flow-insensitive case or in $O(n^4)$ in the flow-sensitive case.
%
Using more advanced forms of polyvariance or context-sensitivity, the analysis becomes significantly more expensive.


In this paper, we describe our ongoing work to encode these analyses as declarative datalog programs and
implement them as highly parallel relational algebra (RA), on the GPU and across many CPUs.
%
Relational algebra operations, derived from Datalog analysis specifications, are computationally intensive and memory-bound in nature.
%
GPUs provide massive fine-grained parallelism and great memory bandwidth, potentially making them the ideal
target for solving these operations. We use Redfox~\cite{Wu:2014:RFE:2581122.2544166}, an open source tool which 
executes queries expressed in a specialized query language on GPUs. We also modify Redfox, adding
capabilities to perform fixed-point iterations essential for solving RA operations derived from Datalog.
%
We are also pursuing a Message Passing Interface (MPI)-based backend for solving RA operations across many compute nodes on a network. 
This approach is also particularly promising, given that HPC is increasingly mainstream and supercomputers are getting
faster cores and lower latency interconnects.


